\documentclass[11pt,letter]{amsart}
\usepackage[utf8]{inputenc}
\usepackage{graphicx}
\usepackage{amssymb}
\usepackage{amsmath}
\usepackage{epstopdf}
\usepackage{amsthm}

\usepackage{enumerate}
%\usepackage{sidenotes}

\usepackage{fourier}
%\usepackage{a4wide}


\newtheorem{definition}{Definition} 
\newtheorem{proposition}[definition]{Proposition} 
\newtheorem{theorem}[definition]{Theorem} 
\newtheorem{lemma}[definition]{Lemma} 
\newtheorem{corollary}[definition]{Corollary}
\newtheorem{conjecture}[definition]{Conjecture}
\newtheorem{question}[definition]{Question}
\newtheorem{claim}[definition]{Claim}
\newtheorem{fact}[definition]{Fact}
\newtheorem{remark}[definition]{Remark}
\newtheorem{wedgelemma}[definition]{Wedge Lemma} 
\newtheorem{construction}[definition]{Construction}\newtheorem{example}[definition]{Example}

\newcommand{\lo}{\preceq}
\newcommand{\rar}{\rightarrow}
\newcommand{\rAr}{\Rightarrow}
\newcommand{\lrAr}{\Leftrightarrow}
\newcommand{\RR}{\mathbb{R}}
\newcommand{\NN}{\mathbb{N}}
\newcommand{\QQ}{\mathbb{Q}}
\newcommand{\ZZ}{\mathbb{Z}}
\newcommand{\HHH}{\mathcal{H}}
\newcommand{\AAA}{\mathcal{A}}
\newcommand{\K}{\mathbf{K}}
\renewcommand{\L}{\mathbf{L}}
\newcommand{\U}{\mathbf{U}}
\newcommand{\V}{\mathbf{V}}
\newcommand{\X}{\mathbf{X}}
\newcommand{\Rat}{\operatorname{Rat}}
\newcommand{\ehr}{\operatorname{ehr}}
\newcommand{\dist}{\mathsf{dist}}
\newcommand{\sgn}{\mathrm{sgn}}
\newcommand{\sprod}[2]{\langle #1, #2 \rangle}
\newcommand{\mspan}{\operatorname{span}}
\renewcommand{\dim}{\mathsf{dim}\ }
\newcommand{\inter}{\mathsf{int}\ }
\newcommand{\median}{\mathrm{median}\ }
\DeclareMathOperator*{\argmin}{arg\,min}

\newcommand{\defn}[1]{\emph{#1}}
\newcommand{\norm}[1]{|| #1 ||}

\newcommand{\Glat}{G^{\operatorname{lat}}}
\newcommand{\Gdep}{G^{\operatorname{dep}}}
\newcommand{\Gfine}{G^{\operatorname{fine}}}
\newcommand{\Gfinem}{G^{\operatorname{fine*}}}

\newcommand{\FD}{\mathcal{F}}
\newcommand{\path}{\operatorname{path}}
\newcommand{\reldep}[2]{{#1} \longrightarrow {#2}}
\newcommand{\reldepi}[3]{{#1} \overset{#2}{\longrightarrow} {#3}}
\newcommand{\shift}[2]{\rightarrow(#1,{#2})}
\newcommand{\shiftw}[3]{\overset{#1 \cdot #2}{\longrightarrow} {#3}}

\newcommand{\conv}{\operatorname{conv}}
\newcommand{\cone}{\operatorname{cone}}
\newcommand{\lev}{\operatorname{lev}}
\newcommand{\Lev}{\operatorname{Lev}}
\renewcommand{\deg}{\operatorname{deg}}
\renewcommand{\dim}{\operatorname{dim}}
\renewcommand{\min}{\operatorname{min}}
\renewcommand{\max}{\operatorname{max}}
\newcommand{\fract}{\operatorname{frac}}
\newcommand{\integ}{\operatorname{int}}
\newcommand{\relint}{\operatorname{relint}}

\newcommand{\twin}{\mathsf{twin}}
\newcommand{\hull}{\mathsf{hull}}
\newcommand{\diam}{\mathsf{diam}}
\newcommand{\choice}[1]{\left\{ \begin{array}{ll} #1 \end{array} \right.}
\newcommand{\floor}[1]{\lfloor {#1} \rfloor}
\newcommand{\ceil}[1]{\lceil {#1} \rceil}
\newcommand{\mset}[2]{ \left\{ #1 \; \middle| \; #2 \right\}}
\newcommand{\lk}{\mathsf{lk}}
\newcommand{\sd}{\mathsf{sd}}
\newcommand{\Pa}{\mathrm{P}}
\newcommand{\dotcup}{\ensuremath{\mathaccent\cdot\cup}}
\newcommand{\Hom}{\mathrm{ Hom}}


\title{Notes}
\author{Felix Breuer, Caroline J. Klivans}
\begin{document}
\maketitle


%%%%%%%%%%%%%%%%%%%%%%%%%%%%%%%%%
\section{Outline}
%%%%%%%%%%%%%%%%%%%%%%%%%%%%%%%%%

\begin{enumerate}
\item Flows and Hyperplane Arrangement Duality
\item Hopf Structure on Zonotopes/Hyperplane Arrangements in General
\item Scheduling Problems, Quasisymmetric Functions and Polynomials
\end{enumerate}


%%%%%%%%%%%%%%%%%%%%%%%%%%%%%%%%%
\section{Hyperplane Arrangement Duality}
%%%%%%%%%%%%%%%%%%%%%%%%%%%%%%%%%

Let $V\subset \ZZ^d\setminus\{0\}$ be a finite set of primitive integer vectors.\footnote{Primitive means that the $\gcd$ of the entries of each vector is 1.} Let $\HHH$ denote the central hyperplane arrangement
\[
\HHH = \mset{H_v}{v\in V}
\]
where
\[
H_v = \mset{x\in \RR^d}{\sprod{x}{v}=0}.
\]
The cones in $\HHH$ are rational, whence their generators can be chosen to be primitive and integral. Let $G(V)$ denote the set of primitive integral generators of $\HHH$. In other words, $G(V)$ can be described as follows:
\[
 G(V) = \mset{w\in\ZZ^d\setminus\{0\}\text{ primitive}}{\exists T\subset V: \dim T^\perp = 1 \wedge w\in T^\perp}
\]
where $T^\perp = \mset{x\in\RR^d}{\sprod{x}{y}=0 \text{ for all} y\in T}$.

Note: Defined this way, $G(V)$ always produces a centrally symmetric set of vectors. We may or may not want this convention.

\begin{lemma}
$G$ maps a finite set of primitive lattice points to a finite set of primitive lattice points.
\end{lemma}

\begin{proof}
There are only finitely many subsets $T$ to choose from. If $\dim T^\perp = 1$ and $T$ is rational there are exactly two primitive non-zero lattice points in  $T^\perp$.
\end{proof}

\begin{lemma}
Let $V,W\subset \ZZ^d\setminus\{0\}$ be a finite sets of primitive integer vectors that span $\RR^d$. Then 
\begin{enumerate}
\item $V\subset W$ $\rAr$ $G(V)\subset G(W)$,
\item $V \subset G(G(V))$.
\end{enumerate}
\end{lemma}

\begin{proof}
(1) If $W\supset V$ then in the definition $G(W)$ there are more subsets $T$ to choose from than in the definition of $G(V)$.

(2) Let $v\in V$. Define $v_1:=v$ and let $v_2,\ldots,v_d\in V$ such that $v_1,v_2,\ldots,v_d$ is a basis for $\RR^d$. Let $V_i=\{v_1,\ldots,v_d\}\setminus\{v_i\}$. Then $\dim V_i^\perp = 1$ for every $i$. Moreover, if $w\in V_i^\perp$ for $i=2,\ldots,d$, then $w\perp v$. Let $w_i$ denote a primitive lattice vector in $V_i^\perp$. By construction $w_i\in G(V)$. Moreover, $w_2,\ldots,w_d$ are linearly independent\footnote{What's a good proof for linear independence of the $w_i$?}, so $\dim \{w_2,\ldots,w_d\}^\perp=1$ and $v\in \{w_2,\ldots,w_d\}^\perp$, whence $v\in G(G(V))$ as desired.  
\end{proof}


%%%%%%%%%%%%%%%%%%%%%%%%%%%%%%%%%
\section{Scheduling Problems}
%%%%%%%%%%%%%%%%%%%%%%%%%%%%%%%%%

A \defn{scheduling problem} $S$ on $d$ items is given by a boolean formula $\phi$ over atomic formulas $x_i\leq x_j$ for $i,j\in[d]$. A \defn{$k$-schedule} solving $S$ is a function $x:[d]\rightarrow[k]$ such that $\phi(x)$ is true. The schedule function $\xi_S(k)$ counts the number of $k$-schedules of $S$.

From the standard braid arrangement setup, we get the following results:
\begin{enumerate}
\item $\xi_S(k)$ is a polynomial in $k$.
\item There is a natural quasisymmetric function in non-commuting variables that specializes to $\xi_S$.
\item $\xi_S$ includes the chromatic polynomial of a graph and the "unknown" matroid polynomial as special cases.
\item $\xi_S$ includes Ehrhart polynomials of order polytopes as a special case.
\item The values of $\xi_S(-k)$ at negative integers can be written as a difference of two counting functions. (This gives "one half" of a reciprocity theorem.) This should also be true on the quasisymmetric function level.
\item We get some elementary bounds on the coefficients of $\xi_S$. For example, the $f^*$-vector is non-negative. 
\end{enumerate}

This is of course just the start. We call schedules $S$ \defn{nice} that give rise to a unit cube with a co-dimension 1 subcomplex removed, such that the regions are open full-dim polytopes. Then:
\begin{enumerate}
\item If $S$ is nice, we get an honest reciprocity theorem, both on the quasisymmetric function level as well as on the polynomial level.
\item This case still includes chromatic polynomials and the matroid polynomials.
\item We get strong bounds on the coefficients of the polynomial in the Ehrhart setting (coming from convex ear decompositions).
\end{enumerate}

Of course there are still a bunch of things that are not clear at all, yet:
\begin{enumerate} 
\item Do the convex ear decompositions also tell us something about coefficients on the quasisymmetric function level?
\item Which schedules come from matroids? 
\end{enumerate}


\bibliographystyle{amsplain}
\bibliography{references}

\end{document}
